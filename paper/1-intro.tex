\section{Introduction}\label{sec-intro}

% , further referred to simply as \emph{effects}

Monads, introduced to functional programming by~\citet{1995_wadler_monads}, are
a powerful and general approach for describing effectful (or impure)
computations using pure functions. The key ingredient of the monad abstraction
is the \emph{bind} operator, denoted by \hs{>>=} in
Haskell\footnote{We use Haskell throughout this paper, but the presented results
are not specific to Haskell and do not require any advanced features of the
Glasgow Haskell Compiler (any Haskell98~\citep{haskell98} compliant compiler
will do). Furthermore, we release two libraries for selective applicative
functors along with this paper: for Haskell and OCaml.}:

\vspace{1mm}
\begin{minted}[xleftmargin=10pt]{haskell}
(>>=) :: Monad f => f a -> (a -> f b) -> f b
\end{minted}
\vspace{1mm}

\noindent
The operator takes two arguments: an effectful computation \hs{f}~\hs{a}, which
yields a value of type~\hs{a} when executed, and a recipe, i.e. a pure function
of type \hs{a}~\hs{->}~\hs{f}~\hs{b}, for turning~\hs{a} into a subsequent
computation of type \hs{f}~\hs{b}. This approach to composing effectful
computations is inherently sequential: until you execute the effects in
\hs{f}~\hs{a}, you have no way of obtaining the continuation \hs{f}~\hs{b},
i.e. these computations can only be performed in sequence.

Consider a simple example, where we use the monad \hs{f}~\hs{=}~\hs{IO} to
describe an effectful program that prints \hs{"pong"} if the user enters
\hs{"ping"}:

\vspace{1mm}
\begin{minted}[xleftmargin=10pt]{haskell}
pingPongM :: IO ()
pingPongM = getLine
            >>=
            \s -> if s == "ping" then putStrLn "pong" else pure ()
\end{minted}
\vspace{1mm}

\noindent
Here the first argument of the bind operator reads a string using
\hs{getLine}~\hs{::}~\hs{IO}~\hs{String}, and the second argument is the
function of type \hs{String}~\hs{->}~\hs{IO}~\hs{()}, which examines the string
and prints \hs{"pong"} if need be. While this works, the function is completely
opaque: there is no way to "see through" the lambda \hs{\s}~\hs{->}~\hs{...} to
predict the effects it might perform. Instead of conditionally executing
\hs{putStrLn}, as intended, it could delete a file from disk, or launch
proverbial missiles. As we will see in sections~\S\ref{sec-static}
and~\S\ref{sec-haxl}, in some applications it is desirable to know all possible
effects \emph{statically}, i.e. before starting the computation.

Applicative functors, introduced by~\citet{mcbride2008applicative}, can be used
for composing a statically known collection of effectful computations, as long
as these computations are \emph{independent}. The key ingredient of applicative
functors is the \emph{apply} operator, denoted by \hs{<*>}:

\vspace{1mm}
\begin{minted}[xleftmargin=10pt]{haskell}
(<*>) :: Applicative f => f (a -> b) -> f a -> f b
\end{minted}
\vspace{1mm}

\noindent
The operator takes two effectful computations, which -- independently -- compute
values of types \hs{a}~\hs{->}~\hs{b} and \hs{a}, and returns their composition
that performs both computations, and applies the obtained function to the
obtained value producing the result of type \hs{b}. Crucially, both arguments
and associated effects are known statically, which, for example, allows us to
allocate all necessary resources upfront (\S\ref{sec-static}) or execute both
computations in parallel (\S\ref{sec-haxl}).

Alas, our ping-pong example cannot be expressed using applicative functors.
Since the two computations must be independent, the best we can do is to print
\hs{"pong"} unconditionally:

\vspace{1mm}
\begin{minted}[xleftmargin=10pt]{haskell}
pingPongA :: IO ()
pingPongA = fmap (const id) getLine
            <*>
            putStrLn "pong"
\end{minted}
\vspace{1mm}

\noindent
We use \hs{fmap}~\hs{(}\hs{const}~\hs{id)} to replace the input string, which we
now have no need for, with the identity function to match the return type of
\hs{putStrLn}~\hs{::}~\hs{IO}~\hs{()}.

