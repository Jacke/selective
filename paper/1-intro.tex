\section{Introduction}\label{sec-intro}

% , further referred to simply as \emph{effects}

Monads, introduced to functional programming by~\citet{1995_wadler_monads}, are
a powerful and general approach for describing effectful (or impure)
computations using pure functions. The key ingredient of the monad abstraction
is the \emph{bind} operator, denoted by \hs{>>=} in
Haskell\footnote{We use Haskell throughout this paper, but the presented results
are not specific to Haskell and do not require any advanced features of the
Glasgow Haskell Compiler (any Haskell98~\citep{haskell98} compliant compiler
will do). Furthermore, we release two libraries for selective applicative
functors along with this paper: for Haskell and OCaml.}:

\vspace{1mm}
\begin{minted}[xleftmargin=10pt]{haskell}
(>>=) :: Monad f => f a -> (a -> f b) -> f b
\end{minted}
\vspace{1mm}

\noindent
The operator takes two arguments: an effectful computation \hs{f}~\hs{a}, which
yields a value of type~\hs{a} when executed, and a recipe, i.e. a pure function
of type \hs{a}~\hs{->}~\hs{f}~\hs{a}, for turning~\hs{a} into a subsequent
computation of type \hs{f}~\hs{b}. This approach to composing effectful
computations is inherently sequential: until you execute the effects in
\hs{f}~\hs{a}, you have no way of obtaining the continuation \hs{f}~\hs{b},
i.e. these computations can only be performed in sequence.
