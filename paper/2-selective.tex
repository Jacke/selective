\newpage

\section{Selective functors}\label{sec-selective}

\begin{figure}
\begin{minted}[fontsize=\small]{haskell}
class Functor f where
    fmap :: (a -> b) -> f a -> f b

-- An infix synonym for fmap
(<$>) :: Functor f => (a -> b) -> f a -> f b       -- (<$>) is pronounced as "map"

class Functor f => Applicative f where
    pure  :: a -> f a
    (<*>) :: f (a -> b) -> f a -> f b              -- (<*>) is pronounced as "apply"

-- A variant of <*> with the arguments reversed
(<**>) :: Applicative f => f a -> f (a -> b) -> f b

class Applicative f => Selective f where
    select :: f (Either a b) -> f (a -> b) -> f b

-- An infix synonym for select
(<*?) :: f (Either a b) -> f (a -> b) -> f b       -- (<*?) is pronounced as "select"

class Selective f => Monad f where
    return :: a -> f a
    (>>=)  :: f a -> (a -> f b) -> f b             -- (>>=) is pronounced as "bind"
\end{minted}
\caption{The proposed type class hierarchy, where \hs{Functor}, \hs{Applicative}
and \hs{Monad} are standard Haskell type classes, and \hs{Selective} is
a new intermediate abstraction introduced between \hs{Applicative} and
\hs{Monad}.}\label{fig-types}
\end{figure}

See Fig.~\ref{fig-types}.

It may be illuminating to compare the following type signatures:

\begin{minted}[xleftmargin=10pt]{haskell}
(<**>) :: Applicative f => f a            -> f (a -> b) -> f b
select :: Selective   f => f (Either a b) -> f (a -> b) -> f b
(>>=)  :: Monad       f => f a            -> (a -> f b) -> f b
\end{minted}

\subsection{Basic examples}

Const

Validation
