\section{Selective functors}\label{sec-selective}

\begin{figure}
\begin{minted}[fontsize=\small]{haskell}
class Functor f where
    fmap :: (a -> b) -> f a -> f b
\end{minted}
\vspace{0.5mm}
\begin{minted}[fontsize=\small]{haskell}
-- An infix synonym for fmap
(<$>) :: Functor f => (a -> b) -> f a -> f b       -- (<$>) is pronounced "map"
\end{minted}
\vspace{2mm}
\hrule
\vspace{2mm}
\begin{minted}[fontsize=\small]{haskell}
class Functor f => Applicative f where
    pure  :: a -> f a
    (<*>) :: f (a -> b) -> f a -> f b              -- (<*>) is pronounced "apply"
\end{minted}
\vspace{0.5mm}
\begin{minted}[fontsize=\small]{haskell}
-- A variant of <*> with the arguments reversed
(<**>) :: Applicative f => f a -> f (a -> b) -> f b
\end{minted}
\vspace{0.5mm}
\begin{minted}[fontsize=\small]{haskell}
-- A variant of <*> that discards the value of the first argument
(*>) :: Applicative f => f a -> f b -> f b
\end{minted}
\vspace{2mm}
\hrule
\vspace{2mm}
\begin{minted}[fontsize=\small]{haskell}
class Applicative f => Selective f where
    select :: f (Either a b) -> f (a -> b) -> f b
\end{minted}
\vspace{0.5mm}
\begin{minted}[fontsize=\small]{haskell}
-- An infix synonym for select
(<*?) :: f (Either a b) -> f (a -> b) -> f b       -- (<*?) is pronounced "select"
\end{minted}
\vspace{2mm}
\hrule
\vspace{2mm}
\begin{minted}[fontsize=\small]{haskell}
class Selective f => Monad f where
    return :: a -> f a                             -- Note the law: return = pure
    (>>=)  :: f a -> (a -> f b) -> f b             -- (>>=) is pronounced "bind"
\end{minted}
\vspace{0.5mm}
\begin{minted}[fontsize=\small]{haskell}
-- A monadic equivalent of the apply operator, satisfying the law (<*>) = ap
@\blk{ap}@ :: Monad f => f (a -> b) -> f a -> f b
\end{minted}
\caption{The proposed type class hierarchy, where \hs{Functor}, \hs{Applicative}
and \hs{Monad} are standard Haskell type classes, and \hs{Selective} is
a new intermediate abstraction introduced between \hs{Applicative} and
\hs{Monad}.}\label{fig-types}
\vspace{-2mm}
\end{figure}

In this section we introduce selective applicative functors, which we will
further refer to as simply \emph{selective functors}, for brevity. We start by
defining the new abstraction, and then use it in~\S\ref{sec-combinators} to
implement several simple combinators, such as the aforementioned \hs{whenS}.
In~\S\ref{sec-instances} we provide several examples of selective functors, and
discuss the relationships between applicative functors, selective functors, and
monads. In~\S\ref{sec-laws}, these relationships are further elaborated and
expressed as a set of laws that all selective functors are required to satisfy.

Like applicative functors~\citep{mcbride2008applicative}, selective functors
provide a way to embed pure values into an effectful context \hs{f} using the
function \hs{pure}, and give meaning to composition of two \emph{independent}
effectful computations using the operator \hs{<*>}. See Fig.~\ref{fig-types} for
the standard definition of the corresponding type class \hs{Applicative}.
Selective functors enrich the applicative interface with the \hs{select}
function, which gives meaning to the composition of two effectful computations,
where, in contrast to \hs{<*>}, the second computation \emph{depends} on the
first one:

\vspace{1mm}
\begin{minted}[xleftmargin=10pt]{haskell}
class Applicative f => Selective f where
    select :: f (Either a b) -> f (a -> b) -> f b
\end{minted}
\vspace{1mm}

\noindent
One can think of \hs{select} as a selective function application:
parametricity~\citep{wadler1989theorems} dictates that, when given a
\hs{Left}~\hs{a}, we \emph{must execute the effects} in
\hs{f}~\hs{(}\hs{a}~\hs{->}~\hs{b)}, apply the obtained function to \hs{a}, and
return the resulting \hs{b}; on the other hand, when given a \hs{Right}~\hs{b},
we \emph{may skip the effects} associated with the function, and return the
given \hs{b}.

Following the notational convention for applicative operators, we also define
the infix operator alias \hs{<*?} for \hs{select}: the angle bracket pointing to
the left means we always use the corresponding value; the value on the right,
however, may be skipped, hence the question mark.

\begin{figure}
\begin{minted}[fontsize=\small]{haskell}
($)     :: (a -> b) -> a -> b                               -- Function application
(.)     :: (b -> c) -> (a -> b) -> a -> c                   -- Function composition
id      :: a -> a                                           -- Identity function
const   :: a -> b -> a                                      -- Constant function
flip    :: (a -> b -> c) -> b -> a -> c                     -- Flip function arguments
uncurry :: (a -> b -> c) -> (a, b) -> c                     -- Uncurry a function
foldr   :: (a -> b -> b) -> b -> [a] -> b                   -- Reduce a list to a value
bool    :: a -> a -> Bool -> a                              -- Deconstruct a Bool
maybe   :: b -> (a -> b) -> Maybe a -> b                    -- Deconstruct a Maybe
either  :: (a -> c) -> (b -> c) -> Either a b -> c          -- Deconstruct a Either
first   :: (a -> c) -> Either a b -> Either c b             -- Map over Left
second  :: (b -> c) -> Either a b -> Either a c             -- Map over Right
bimap   :: (a -> c) -> (b -> d) -> Either a b -> Either c d -- Map over Left and Right
void    :: Functor f => f a -> f ()                         -- Discard an effect's value
\end{minted}
\caption{Type signatures of standard operators and functions used throughout the
paper.}\label{fig-std}
\end{figure}

One can implement a function with the type signature of \hs{select} using monads
in a straightforward manner: examine the value produced by
\hs{f}~\hs{(Either}~\hs{a}~\hs{b)} with the bind operator, and then, in the
\hs{Left}~\hs{a} case, execute the subsequent effect, passing the \hs{a} to it
using the \hs{Functor}'s map operator, as shown below (we use \hs{<@\fmap@>}, an
infix alias of \hs{fmap}, see Fig.~\ref{fig-types} for the type signatures):

\vspace{1mm}
\begin{minted}[xleftmargin=10pt]{haskell}
selectM :: Monad f => f (Either a b) -> f (a -> b) -> f b
selectM x y = x >>= \e -> case e of Left  a -> ($a) <$> y -- execute y
                                    Right b -> pure b     -- skip y
\end{minted}
\vspace{1mm}

\noindent
Many monads directly use \hs{select}~\hs{=}~\hs{selectM} in their \hs{Selective}
instance definitions, and in \S\ref{sec-laws} we argue that this should in fact
be a law when both \hs{Selective}~\hs{f} and \hs{Monad}~\hs{f} instances exist.
Note that some monads, e.g. the \Haxl monad (\S\ref{sec-haxl}), choose to
implement the  \hs{select} method differently for performance reasons, but they
still satisfy the law \hs{select}~\hs{=}~\hs{selectM} at the semantic level.

One can also implement \hs{select} using applicative functors, but it will
always execute the effects associated with the second argument, rendering any
conditional execution of effects impossible, as in the \hs{pingPongA} example
in~\S\ref{sec-intro}:

\vspace{1mm}
\begin{minted}[xleftmargin=10pt]{haskell}
selectA :: Applicative f => f (Either a b) -> f (a -> b) -> f b
selectA x y = (\e f -> either f id e) <$> x <*> y
\end{minted}
\vspace{1mm}

\noindent
Fig.~\ref{fig-std} gives type signatures and short descriptions for standard
functions \hs{either}, \hs{id}, as well as many other convenient functional
combinators used throughout this paper.

While \hs{selectM} is useful for \emph{conditional execution} of effects,
\hs{selectA} is useful for \emph{static analysis}. As we will see
in~\S\ref{sec-instances}, the above definition is sometimes useful, for example,
selective functors used for static analysis need to collect information about
all possible effects instead of skipping some of them, hence directly using
\hs{select}~\hs{=}~\hs{selectA} in their \hs{Selective} instance definitions.

Any \hs{Applicative} instance can thus be given a \hs{Selective} instance. The
opposite is also true in the sense that one can recover the operator \hs{<*>}
from \hs{select} as follows:

\vspace{1mm}
\begin{minted}[xleftmargin=10pt]{haskell}
apS :: Selective f => f (a -> b) -> f a -> f b
apS f x = select (Left <$> f) (flip ($) <$> x)
\end{minted}
\vspace{1mm}

\noindent
Here we tag a given function \hs{a}~\hs{->}~\hs{b} with \hs{Left} and turn a
value of type \hs{a} into the reverse application function
\hs{\}\hs{a}~\hs{f}~\hs{->}~\hs{f}~\hs{a}, which yields \hs{b} when given
\hs{a}~\hs{->}~\hs{b}, as desired. Since the \hs{Right} case is impossible, the
effect \hs{f}~\hs{a} is executed unconditionally. Note however, that the
equality \hs{(<*>)}~\hs{=}~\hs{apS} does not always hold. Selective functors
that satisfy the law \hs{(<*>)}~\hs{=}~\hs{apS} will be called \emph{rigid};
they will turn out to have a particularly simple normal form, which we will
exploit in the free construction in~\S\ref{sec-free}.

We will come back to the relationship between applicative functors, selective
functors and monads in \S\ref{sec-laws}, after first exploring \emph{selective
combinators} that can be written using the selective interface
(\S\ref{sec-combinators}), and looking at some concrete examples of selective
functors (\S\ref{sec-instances}).

\subsection{Selective combinators}\label{sec-combinators}

As a first use-case of the interface provided by selective functors, let us
revisit our ping-pong example from~\S\ref{sec-intro} and implement the
combinator \hs{whenS}:

\vspace{1mm}
\begin{minted}[xleftmargin=10pt]{haskell}
whenS :: Selective f => f Bool -> f () -> f ()
whenS x y = selector <*? effect
  where
    selector = bool (Right ()) (Left ()) <$> x -- NB: maps True to Left ()
    effect   = const                     <$> y
\end{minted}
\vspace{1mm}

\noindent
We first bring the given effectful computations into the right shape by using
the \hs{Functor}'s function \hs{fmap} (see
Fig.~\ref{fig-types}); specifically, \hs{x}~\hs{::}~\hs{f}~\hs{Bool} is
converted into a \hs{selector}~\hs{::}~\hs{f}~\hs{(Either}~\hs{()}~\hs{())}, and
\hs{y}~\hs{::}~\hs{f}~\hs{()} is converted into
\hs{effect}~\hs{::}~\hs{f}~\hs{(()}~\hs{->}~\hs{())}. The results are composed
using the select operator \hs{<*?}, and the meaning of this composition is
determined by the supplied \hs{Selective}~\hs{f} instance. For example, an instance
like \hs{f}~\hs{=}~\hs{IO} would skip \hs{y} if \hs{x} yields \hs{True}, as
exploited by our implementation of \hs{pingPongS}. On the other hand, instances
used for static analysis would record both \hs{x} and \hs{y} as possible
effects. See more examples in~\S\ref{sec-instances}.

It is worth noting that unlike the select operator, whose implementation is
almost completely determined by parametricity (i.e., the only real question is:
\emph{"To skip, or not to skip?"}), \hs{whenS} admits a variety of (incorrect)
implementations. In particular, due to \emph{Boolean blindness}\footnote{The
term refers to the fact that the \hs{True} and \hs{False} values are not
distinguished at the type level, see~\citet{boolean-blindness}.},
it is easy to inadvertently implement \hs{unlessS}, which has the same type but
flips the meaning of the Boolean value. The ability to reason parametrically was
one of the guiding principles we used when looking for a good abstraction for
selective functors: \hs{select} provides this ability, whereas \hs{whenS} does
not.

% Constraints liberate, liberties constrain
% Dijkstra: "Being abstract is something profoundly different from being vague …
% The purpose of abstraction is not to be vague, but to create a new semantic
% level in which one can be absolutely precise."

A strong contender for playing the first violin in selective functors is the
function \hs{branch} that, given an effectful computation
\hs{x}~\hs{::}~\hs{f}~\hs{(Either}~\hs{a}~\hs{b)}, selects which subsequent
computation, namely \hs{l}~\hs{::}~\hs{f}~\hs{(}\hs{a}~\hs{->}~\hs{c)} or
\hs{r}~\hs{::}~\hs{f}~\hs{(}\hs{b}~\hs{->}~\hs{c)}, to execute:

\vspace{1mm}
\begin{minted}[xleftmargin=10pt]{haskell}
branch :: Selective f => f (Either a b) -> f (a -> c) -> f (b -> c) -> f c
branch x l r = fmap (fmap Left) x <*? fmap (fmap Right) l <*? r
\end{minted}
\vspace{1mm}

\begin{figure}
\begin{minted}[fontsize=\small]{haskell}
branch :: Selective f => f (Either a b) -> f (a -> c) -> f (b -> c) -> f c
branch x l r = fmap (fmap Left) x <*? fmap (fmap Right) l <*? r

ifS :: Selective f => f Bool -> f a -> f a -> f a
ifS x t e = branch (bool (Right ()) (Left ()) <$> x) (const <$> t) (const <$> e)

whenS :: Selective f => f Bool -> f () -> f ()
whenS x y = ifS x y (pure ())

whileS :: Selective f => f Bool -> f () -- Run a computation while it yields True
whileS x = whenS x (whileS x)

(<||>) :: Selective f => f Bool -> f Bool -> f Bool
x <||> y = ifS x (pure True) y

(<&&>) :: Selective f => f Bool -> f Bool -> f Bool
x <&&> y = ifS x y (pure False)

anyS :: Selective f => (a -> f Bool) -> [a] -> f Bool
anyS p = foldr ((<||>) . p) (pure False)

allS :: Selective f => (a -> f Bool) -> [a] -> f Bool
allS p = foldr ((<&&>) . p) (pure True)

fromMaybeS :: Selective f => f a -> f (Maybe a) -> f a
fromMaybeS x mx = select (maybe (Left ()) Right <$> mx) (const <$> x)
\end{minted}
\caption{A library of selective combinators. The names and order of parameters
are inherited from the standard Haskell library. For example, \hs{fromMaybeS}
corresponds to the standard
\hs{fromMaybe}~\hs{::}~\hs{a}~\hs{->}~\hs{Maybe}~\hs{a}~\hs{->}~\hs{a} and
retains the short-circuiting behaviour, i.e. if the second argument is a
\hs{Just}, the first argument is skipped.}
\label{fig-library}
\end{figure}

% Add `orElse` and `andAlso' combinators?

\noindent
While we encourage the reader to derive an implementation of \hs{branch} as an
exercise, we would like to share our intuition behind it, as it will be useful
for \emph{free selective functors} in~\S\ref{sec-free}. The select operator
allows us to eliminate one of the cases in a sum type, namely the
\hs{Left}~\hs{a} case in \hs{Either}~\hs{a}~\hs{b}, leaving the other case
intact. To implement \hs{branch}, we will need to apply \hs{<*?} twice,
eliminating \hs{a} and \hs{b} one after another. The first application is tricky
because \hs{f}~\hs{(Either}~\hs{a}~\hs{b)} and
\hs{f}~\hs{(}\hs{a}~\hs{->}~\hs{c)} do not match the type signature of \hs{<*?}.
To fix the mismatch, we convert them to
\hs{f}~\hs{(Either}~\hs{a}~\hs{(}\hs{Either}~\hs{b}~\hs{c))} and
\hs{f}~\hs{(}\hs{a}~\hs{->}~\hs{Either}~\hs{b}~\hs{c)}, respectively. The second
application of \hs{<*?} is then straightforward.

As will be discussed in~\S\ref{sec-alternatives}, we could have chosen to use
\hs{branch} instead of \hs{select} as the method of the \hs{Selective} type
class. Our choice of \hs{select} follows the Occam's razor principle:
\hs{select} is simpler than \hs{branch}, which, in particular, leads to a
simpler free construction (\S\ref{sec-free}).

By instantiating \hs{select} with \hs{a}~\hs{=}~\hs{b}~\hs{=}~\hs{()} we have
earlier obtained \hs{whenS}. Below we repeat the same trick but with
\hs{branch}, obtaining another familiar conditional combinator \hs{ifS}:

\vspace{1mm}
\begin{minted}[xleftmargin=10pt]{haskell}
ifS :: Selective f => f Bool -> f a -> f a -> f a
ifS x t e = branch selector (fmap const t) (fmap const e)
  where
    selector = fmap (\b -> if b then Left () else Right ()) x
\end{minted}
\vspace{1mm}

\noindent
Many conditional combinators, which are typically associated with the \hs{Monad}
type class, can be expressed using selective functors, as demonstrated in
Fig.~\ref{fig-library}, making them reusable in new contexts. In particular, the
logical combinators \hs{<||>} and \hs{<&&>} will play an important role in
improving the efficiency of the \Haxl framework in~\S\ref{sec-haxl}.

\subsection{Basic examples}\label{sec-instances}

Over, under

Validation

\subsection{Laws}\label{sec-laws}

It may be illuminating to compare the following type signatures:

\begin{minted}[xleftmargin=10pt]{haskell}
(<**>) :: Applicative f => f a            -> f (a -> b) -> f b
select :: Selective   f => f (Either a b) -> f (a -> b) -> f b
(>>=)  :: Monad       f => f a            -> (a -> f b) -> f b
\end{minted}

The type signature of \hs{select} is reminiscent of both \hs{<*>} and \hs{>>=},
and indeed a selective functor is in some sense a composition of an applicative
functor and the \hs{Either} monad.

\begin{figure}
\begin{minted}{haskell}
------------------------------------- Laws: ------------------------------------
-- Identity
@\blk{x}@ <*? pure id = either id id <$> x

-- Interchange
@\blk{x}@ *> (y <*? z) = (x *> y) <*? z

-- Associativity
@\blk{x}@ <*? (y <*? z) = (f <$> x) <*? (g <$> y) <*? (h <$> z)
  where
    f x = Right <$> x
    g y = \a -> bimap (,a) ($a) y
    h z = uncurry z

-- For selective functors that are also monads:
select = selectM
---------------------------------- Theorems: -----------------------------------
-- Apply a pure function to the result:
f <$> select x y = select (second f <$> x) ((f .) <$> y)

-- Apply a pure function to the Left case of the first argument:
select (first f <$> x) y = select x ((. f) <$> y)

-- Apply a pure function to the second argument:
select x (f <$> y) = select (first (flip f) <$> x) (flip ($) <$> y)

-- General interchange
x <*> (y <*? z) = (f <$> x <*> y) <*? (g <$> z)
  where
    f ab = bimap (\c ca -> ab (ca c)) ab
    g ca = ($ca)

-- For selective functors that are also monads:
(<*>) = apS
\end{minted}
\caption{Laws and theorems of selective functors.}
\label{fig-laws}
\end{figure}

Note that it is not a requirement for selective functors to skip unnecessary
effects. It may be counterintuitive, but this makes them more useful. Why?
Typically, when executing a selective computation, you would want to skip the
effects (saving work); but on the other hand, if your goal is to statically
analyse a given selective computation and extract the set of all possible
effects (without actually executing them), then you do not want to skip any
effects, because that defeats the purpose of static analysis.

...

