\section{Alternative formulations}\label{sec-alternatives}

One might argue that the \hs{select} method should simply be added to the
\hs{Applicative} type class, with the default implementation
\hs{select}~\hs{=}~\hs{selectA}. This is indeed a viable approach, but it has
two drawbacks: (i) this is a breaking change, since the new method will conflict
with any existing definitions of \hs{select}, and (ii) perhaps more importantly,
it will make it harder to reason about code with the \hs{Applicative}~\hs{f}
constraint, since the new method will make it possible for "new applicative"
effects to depend on values. We therefore think that selective applicative
functors deserve a separate type class.

