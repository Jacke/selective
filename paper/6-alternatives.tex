\section{Alternative formulations for selective functors}
\label{sec-alternatives}

This section discusses alternative versions of the \hs{Selective} type class,
which are based on multi-way generalisations of the \hs{select} operator
(\S\ref{sec-alt-multi}), as well as on making it more
symmetric~(\S\ref{sec-alt-symmetric}). Both of these ideas can be readily
integrated into the presented definition of the \hs{Selective} type class by
extending it with new methods and adding new laws that ensure that the
new methods interact with \hs{select} in an appropriate manner. This is common
in standard Haskell libraries, where type classes \hs{Applicative} and
\hs{Monad} include methods like \hs{*>} and \hs{>>} for performance reasons.

Another alternative, which is worth a remark, is to simply move \hs{select}
to the \hs{Applicative} type class, with the default implementation
\hs{select}~\hs{=}~\hs{selectA}. While this works for the purposes discussed in
this paper, it would make it harder to reason about code with the
\hs{Applicative}~\hs{f} constraint, since the \hs{select} method makes it
possible for effects to depend on values; declaring such a significant ability
by the \hs{Selective}~\hs{f} constraint is arguably a more prudent approach.

\subsection{Multiway select operators}\label{sec-alt-multi}

As mentioned in~\S\ref{sec-selective}, \hs{branch} is a strong contender to be
the main method of the \hs{Selective} type class: it is parametric and all
selective combinators, including \hs{select} itself, can be derived from it:

\vspace{1mm}
\begin{minted}[xleftmargin=10pt]{haskell}
branch :: Selective f => f (Either a b) -> f (a -> c) -> f (b -> c) -> f c

selectB :: Selective f => f (Either a b) -> f (a -> b) -> f b
selectB x y = branch x y (pure id)
\end{minted}
\vspace{1mm}

\noindent
While we prefer \hs{select} for its simplicity, \hs{branch} does provide an
interesting advantage in the context of static analysis. Specifically, it makes
it statically apparent that the two branches are \emph{mutually exclusive}. When
\hs{branch} is ``desugared'' into a sequence of two \hs{select} operations, the
information about the mutual exclusion between the two branches is lost, which
rules out some static analysis scenarios. For example, it may be useful to know
that in our build systems example in~\S\ref{sec-static-example} we never depend
on both \cmd{lib.c} and \cmd{lib.ml}.

Another point in favour of \hs{branch} is performance: the \hs{select}-based
implementation of the \hs{ifS} combinator checks for the \hs{Left} and
\hs{Right} cases in sequence, instead of directly jumping to the correct case,
so a \hs{branch}-based implementation would be more efficient. Furthermore,
$N$-way generalisations of \hs{select} are possible, although the design space
here is quite large. As an example, one might consider adding \hs{bindS} to the
\hs{Selective} type class, i.e. a special case of the monadic bind operator that
is applicable only to enumerable types:

\vspace{1mm}
\begin{minted}[xleftmargin=10pt]{haskell}
bindS :: Selective f => (Bounded a, Enum a, Eq a) => f a -> (a -> f b) -> f b
\end{minted}
\vspace{1mm}

\noindent
The default implementation could be based on sequentially checking for every
possible value using \hs{select}, but monadic instances would supply a much
faster implementation, namely \hs{bindS}~\hs{=}~\hs{(>>=)}. This would allow
static analysis instances to record all possible cases, without incurring
the $O(N)$ slowdown during the execution of an $N$-way branch.

Interestingly, adding the ability to branch on infinite number of cases makes
selective functors equivalent to monads, although it is worth pointing out that
static analysis of such infinitely-branching selective functors might take
infinite time too.

Investigation of the design space for ``multiway selective functors'', and
exploiting them for efficient translation of Haskell's \cmd{do}-notation into
selective combinators in the spirit of the \cmd{ApplicativeDo}
extension~\citep{marlow2016applicativedo} is left for future work. For now, we
believe that adding \hs{branch} and/or an equivalent of \hs{bindS} to the
\hs{Selective} type class would be beneficial for performance-sensitive
applications.

\subsection{Symmetric select operators}\label{sec-alt-symmetric}

In this section we address the asymmetry of \hs{select} that we remarked on
in~\S\ref{sec-haxl}. The asymmetry can be seen in the fact that the first
argument of \hs{select} must always be executed, while the second argument may
sometimes be skipped. Consider the following more symmetric alternative:

\begin{minted}[xleftmargin=5pt]{haskell}
biselect :: Selective f => f@\,@(Either a b) -> f@\,@(Either a c) -> f@\,@(Either a (b,c))
\end{minted}

\noindent
This definition is pleasantly symmetric: if either of the arguments yields a
\hs{Left}~\hs{a} value, the other argument \emph{may be skipped} since the
result must be a \hs{Left}~\hs{a} too, by parametricity. On the other hand, if
one of the arguments yields a \hs{Right} value, then the other argument
\emph{must be executed} in order to either get an \hs{a} or the other half of
the resulting pair. As an added bonus, the rather obscure associativity law
from~\S\ref{sec-laws} looks much more natural for
\hs{(?*?)}~\hs{=}~\hs{biselect}:

\vspace{0.5mm}
\begin{minted}[xleftmargin=10pt]{haskell}
@\blk{x}@ ?*? (y ?*? z) = second assoc <$> ((x ?*? y) ?*? z)
  where
    assoc ((a, b), c) = (a, (b, c))
\end{minted}
\vspace{0.5mm}

\noindent
While beautiful, we found \hs{biselect} to be a bit more awkward to work with
than \hs{select}, and also more subtle when the order in which the arguments
are executed is not fixed. So far we have identified only one example where the
symmetry of \hs{biselect} is beneficial: speculative execution of parallel OR
and AND combinators --- see the \Haxl case study~\S\ref{sec-haxl}. To support
such use-cases it is possible to add \hs{biselect} to the \hs{Selective} type
class with the following default implementation:

\vspace{0.5mm}
\begin{minted}[xleftmargin=5pt]{haskell}
biselect :: Selective f => f@\,@(Either a b) -> f@\,@(Either a c) -> f@\,@(Either a (b,c))
biselect x y = select ((fmap Left . swap) <$> x) ((\e a -> fmap (a,) e) <$> y)
  where
    swap = either Right Left -- Swap Left and Right
\end{minted}
\vspace{0.5mm}

\noindent
This implementation breaks the symmetry, which may be acceptable for most
instances of selective functors, but instances like \hs{Haxl} would override it
in order to gain additional performance benefits. Note that the selective
combinators like \hs{<||>} would need to be redefined via \hs{biselect} in order
to take advantage of the symmetry.

From the theoretical viewpoint, the type signature of \hs{biselect} makes it
more apparent that a selective functor \hs{f} is a composition of an applicative
functor \hs{f} and the \hs{Either} monad.

% Arseniy's raise/catch

% Loops
