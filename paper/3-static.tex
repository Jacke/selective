\section{Static analysis}\label{sec-static}

Using build systems as example

\subsection{Dune build system}

Dune was originally developed at Jane Street and has by now become the standard
build system for OCaml packages. At the time of writing, more than 1000 OCaml
packages are using Dune as the build system. The original motivation for
developing Dune (which was earlier known as \textsf{jbuilder}) was to make it
easier to open source code developed in an industrial environment, and so Dune
was not meant to be used for everyday software development. However, Dune's
ability to extract maximum parallelism from build scripts meant it was faster
than existing build systems, such as OCamlbuild, and it quickly became very
popular, with major projects switching to Dune, for example, the Coq proof
assistant [ref].

% TODO: Can we get any performance improvement figures on how much more
% parallelism can be gained through static analysis?

One unusual feature of Dune is the ability to statically approximate all build
dependencies of a package instead of requiring the programmer to manually list
them in a package manifest file.

Package manifest files are typically consumed by package managers like
OPAM [ref] for the purpose of downloading and installing all required
dependencies.

% There are also optional package dependencies. Shall we give more details?

The original aim of this feature was to automatically generate package manifest
files, so that they do not need to be maintained. An alternative approach would
be to integrate the build system with the package manager itself, i.e. whenever
the build system discovers a new external dependency, the package manager would
download and install it, temporarily suspending the build.

The reason Dune could not follow this approach is because package managers are
typically designed to be build system agnostic.

Therefore, the only way to generate the manifest file automatically is to
analyse the build graph statically, i.e. \emph{without actually running any
build commands}, because at this point the project cannot yet be built (due
to missing dependencies). Package dependencies can be conditional and depend
on values that can only be computed during the build, therefore in many
situations it is impossible to statically compute an accurate list of
dependencies, and an over-approximation needs to be computed instead.

In general, one can view this static analysis as a function from a build script
to a list of external dependencies. One way to implement such function would be
to parse the script and extract all possible dependencies.

Another approach, which is adopted in Dune, is to reuse the existing script
execution engine that runs the build commands, but in a mock environment where
commands are skipped, but their dependencies are recorded in all branches of
conditional statements. By doing static analysis at this level, one can reuse
a lot of code, e.g. for parsing and interpreting build scripts, giving us good
confidence in these parts of the implementation.

In this mock environment, some part of the code cannot be fully
evaluated as they need the output produced by external
commands. However, these parts still need to be analysed. To achieve
this, Dune uses an arrow data structure. The reason Dune uses an arrow
rather than an applicative is discussed later, however applicatives
would be suiitable as well in this context. In particular, a
successful attempt was made at replacing arrows by applicatives in
Dune in the past.

Dune is another example where the extra power provided by selective
functions is relevant. To understand why, let's consider the following
example: a user wants to use some optimized C function if it is
available on the system, and fallback to an OCaml implementation if
not. The C and OCaml implementations might have different external
dependencies. Such a test is dynamic since it depends on the system
the users builds the software on. During compilation, we only want to
follow one of the branch, as we clearly don't want to build
implementation only to keep a single one. However, during dependency
analysis for the project manifest we need to scan both branches. The
dependencies discovered in both branches will be considered as
optional dependencies given that neither is always required.

\subsection{Static analysis of dependencies}



Applicative non-implementation
Monadic non-implementation

Selective implementation
Arrow implementation

Discussion.

For parallelism use underapproximation: take intersection of optional
dependencies, and then union with necessary dependencies.
