\section{Related work}\label{sec-related}

Composing effectful computations is a rich research area and there is a vast
body of related work. We build on the fundamental notions of applicative
functors~\citep{mcbride2008applicative} and
monads~\citep{moggi1991notions,1995_wadler_monads}, but these notions are not
isolated: the space between them is inhabited by
\emph{arrows}~\citep{hughes2000arrows} and \emph{generalised
arrows}~\citep{megacz2011hardware}, which we discuss in~\S\ref{sec-arrows}.

The idea of extending the \hs{Applicative} interface to gain more expressive
power is of course not new, and has been particularly popular in
the area of \emph{applicative parser combinators}. In particular,
\citet{swierstra1996parsers} used an extended version of applicative functors
that was later abstracted into the \hs{Alternative} type class, which is
discussed in~\S\ref{sec-alternative-functors}. Other extensions include
combinators for \emph{observable recursion}, e.g.
see~\citep{devriese2012finally,devriese2013fixing}. Combining the latter with
selective functors would give us the ability to statically analyse effectful
computations with cycles.

Our free construction for rigid selective functors~(\S\ref{sec-free}) is
inspired by the works on free applicative functors~\citep{free-applicatives},
free monads~\citep{swierstra2008data}, and insightful blog posts
by~\citet{fancher2016free,fancher2017static}. \emph{Batching and remote
execution} of effectful computations~\citep{gill2015remote} can be greatly
simplified by using free applicative functors, as
demonstrated by~\citet{gibbons2016free}, and we believe that free selective
functors uncover new opportunities in this area.

\subsection{Arrows and profunctors}\label{sec-arrows}

Arrows, introduced by~\citet{hughes2000arrows}, generalise monads by making the
\emph{input} of a computation explicit. Rather than giving the type
\hs{f}~\hs{a} to an effectful computation that yields a value of type~\hs{a}, as
we have done in this paper so far, arrows give the type \hs{a}~\hs{i}~\hs{o} to
an effectful computation that takes values of type \hs{i} as input and yields
values of type \hs{o} as output. There is a rich \emph{arrow hierarchy} of type
classes, each providing a new ability, where \hs{ArrowChoice} is particularly
relevant for us:

\vspace{1mm}
\begin{minted}[xleftmargin=10pt,fontsize=\small]{haskell}
class Category a                  -- Identity arrow, sequential arrow composition
class Category a => Arrow       a -- Pure arrows, parallel arrow composition
class Arrow    a => ArrowChoice a -- Arrows with choice
class Arrow    a => ArrowApply  a -- Arrows that take arrows as input
class Arrow    a => ArrowLoop   a -- Arrows with loops
\end{minted}
\vspace{1mm}

\noindent
The relationships between applicative functors, monads and arrows have been
studied in depth. It is known, e.g. see~\citet{lindley2011idioms}
and~\citet{rivas2017notions}, that applicative functors correspond to so called
\emph{static arrows}, for which there is an isomorphism between
\hs{a}~\hs{()}~\hs{(}\hs{i}~\hs{->}~\hs{o)} and~\hs{a}~\hs{i}~\hs{o}. The
standard module \cmd{Control.Arrow} therefore provides the following
definitions:

\vspace{1mm}
\begin{minted}[xleftmargin=10pt]{haskell}
newtype ArrowMonad a b = ArrowMonad (a () b) -- See Control.Arrow
\end{minted}
\vspace{1mm}
\begin{minted}[xleftmargin=10pt]{haskell}
instance Arrow       a => Functor     (ArrowMonad a)
instance Arrow       a => Applicative (ArrowMonad a)
instance ArrowChoice a => ...                           -- Missing?!
instance ArrowApply  a => Monad       (ArrowMonad a)
\end{minted}
\vspace{1mm}

\noindent
Selective functors provide the missing counterpart for \hs{ArrowChoice} in the
\emph{functor hierarchy}, as demonstrated by the following instance:

\vspace{1mm}
\begin{minted}[xleftmargin=10pt]{haskell}
instance ArrowChoice a => Selective (ArrowMonad a) where
    select (ArrowMonad x) y = ArrowMonad $ x >>> (toArrow y ||| returnA)
\end{minted}
\vspace{1mm}
\begin{minted}[xleftmargin=10pt]{haskell}
toArrow :: Arrow a => ArrowMonad a (b -> c) -> a b c
toArrow (ArrowMonad f) = arr (\x -> ((), x)) >>> first f >>> arr (uncurry ($))
\end{minted}
\vspace{1mm}

\noindent
Here \hs{toArrow} witnesses one half of the aforementioned isomorphism between
\hs{a}~\hs{()}~\hs{(}\hs{i}~\hs{->}~\hs{o)} and~\hs{a}~\hs{i}~\hs{o}. The
obtained \hs{Selective} instance is lawful thanks to the \hs{ArrowChoice} laws.

Arrows are more general and powerful than selective functors. We could have used
arrows to solve our static analysis and speculative execution examples, and not
just in theory --- \Dune is a great example of successful application of arrows
in practice. However, introducing arrows to an existing codebase built around
applicative functors and monads, such as \Haxl, would require pervasive changes
to the whole abstraction stack, as well as rewriting all existing \Haxl user
code in the arrow notation~\citep{paterson2001new}. Needless to say,
introduction of selective functors to \Haxl is a much easier task, which we have
accomplished by adding 13 lines of new code (the definition of the
\hs{Selective}~\hs{Haxl} instance in Fig.~\ref{fig-haxl-select}), and removing
26 lines of code corresponding to similarly-sized definitions of \hs{pOr} and
\hs{pAnd}, reusing the selective combinators \hs{<||>} and \hs{<&&>} instead.

\emph{Profunctors} is an abstraction closely related to arrows;
see~\citep{pickering2017profunctor} for a good overview of profunctors in the
context of modular data accessors, or \emph{lenses}. Similarly to
\hs{ArrowChoice}, so-called \emph{Cocartesian profunctors} are counterparts of
selective functors in the \emph{profunctor hierarchy}.

Establishing a formal correspondence between \hs{ArrowChoice}, Cocartesian
profunctors, and selective functors is beyond the scope of this paper and is
left for future research.

\subsection{Parsers and \hs{Alternative} type class}\label{sec-alternative-functors}

\hs{Alternative} is a type class originally motivated by non-monadic parsers;
see, for example,~\citet{swierstra1996parsers}, where the methods of the
\hs{Alternative} type class appear as part a bigger \hs{Parsing} type class. In
modern Haskell, \hs{Alternative} is a subclass of \hs{Applicative}:

\vspace{1mm}
\begin{minted}[xleftmargin=10pt]{haskell}
class Applicative f => Alternative f where
    empty :: f a
    (<|>) :: f a -> f a -> f a
\end{minted}
\vspace{1mm}

\noindent
The operator \hs{<|>} allows us to naturally express \emph{choice} in parsers.
As an example, consider the task of parsing binary and hexadecimal numbers,
which are prefixed with \hs{"0b"} and \hs{"0x"}, respectively. Following the
classic parser combinator approach~\citep{hutton1998monadic}, let us assume the
existence of the following parsers:

\vspace{1mm}
\begin{minted}[xleftmargin=10pt]{haskell}
sat    :: (Char -> Bool) -> Parser Char   -- Parse a specified character
string :: String         -> Parser String -- Parse a string literal
bin    ::                   Parser Int    -- Parse a binary-encoded number
hex    ::                   Parser Int    -- Parse a hexadecimal-encoded number
\end{minted}
\vspace{1mm}

\noindent
Now the desired parser can be obtained as a choice between parsers for binary
and hexadecimal numbers, each augmented with the prefix-parsing part:

\vspace{1mm}
\begin{minted}[xleftmargin=10pt]{haskell}
numberA :: Parser Int
numberA = (string "0b" *> bin) <|> (string "0x" *> hex)
\end{minted}
\vspace{1mm}

\noindent
When parsing \hs{"0x7E3"}, the first parser fails (due to the prefix mismatch),
but the second one succeeds. Note that parsing of the leading \hs{"0"} can be
factored out into a separate parser \hs{string}~\hs{"0"} to avoid backtracking.

Selective functors also allow us to implement the desired parser, and arguably
in a more direct style that does not involve trying one parser after another:

\vspace{1mm}
\begin{minted}[xleftmargin=10pt]{haskell}
numberS :: Parser Int
numberS = string "0" *> ifS (('b'==) <$> sat (`elem` "bx")) bin hex
\end{minted}
\vspace{1mm}

\noindent
Here we first parse the leading \hs{"0"}, then the second character of the
prefix, failing if it is neither \hs{"b"} nor \hs{"x"}, and finally select an
appropriate subsequent parser using \hs{ifS}. Note that we can move the parser
\hs{string}~\hs{"0"} in and out of the condition \hs{ifS} thanks to the
interchange law (\S\ref{sec-laws}).

Investigation of the relationship between \hs{Alternative} and \hs{Selective}
type classes, as well as application of selective functors to parsers is an
interesting research opportunity.
